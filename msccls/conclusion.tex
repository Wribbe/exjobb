Even though they have old roots, the concepts of user-centered design and
usability-testing still feel fresh and powerful when used as the basis
for a software-development process. And while it was easier to setup the
initial framework for gathering and processing user-submitted test data through
the internet than anticipated. The lack of experience developing a web-based
project of this size combined with choosing a bare-bones web-framework with
little built-support made it impossible to keep up the iteration speed required
for truly benefiting from this type of iterative, design and user-centered
software-development.

In conclusion, it is possible, even for a single developer with some interest,
to kickstart a platform for usability-testing over the internet. And while
having the testing being performed online brings many interesting advantages
with it, there are some extra consideration that need to be considered. Among
them, ensuring that any identifiable data is about the participants is safely
stored and not accessible from the internet. And it also becomes vital to
investigating and deploying procedures to mitigate and catch data
anomalies that are the result of outside influence, such as unstable
internet connections.


%it was easier that expected to
%Did it have an significant impact? Was the web the correct platform? What
%could be done better over the internet? Recording screen and voice?
%(Javascript, since it's already used, pull up some statistics?)
