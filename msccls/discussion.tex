This section reflects on the processes underpinning the report, such as
literary information gathering, the intial design process, the platform
development process, platform deployment, processing-pipeline of the gathered
user data,  together with a broad-view examination on the obtained final
results and any thoughts surrounding those.

\section{Literary information gathering}

  First off, there exists a rich corpus of fascinating literary references
  regarding different flavours of user-centered design, usability and
  usability-testing spanning from present day, back to the 1980s, which might
  make some sources easier to miss. One such missed source is
  \citetitle{citeBeyondTheUsabilityLab}%
  \cite{citeBeyondTheUsabilityLab}. Having access to this reference during the
  initial design and development of the platform would have probably led to a
  different, more literary-grounded design of the questionnaires.

  \todo{Expand on what design changes}

\section{The design process}

  Having the paper prototypes as an initial lo-fi starting point and conducting
  the interviews to determine which to use worked very well, and is an element
  that would be repeated if a similar process would be done in the future.
  That being said, it is regrettable that the iterative design process did not
  have a chance to get into gear and display its full potential over multiple
  design runs. Only managing to get through one
  design-implement-release-and-test-cycle really hamstrung the possibility of
  experience the process of dynamically evolving the platform over time in
  response to user feedback.

\section{The development process}

  It was very easy to set up and get starting working on creating the platform
  with both Python and Flask, and the performance and functionality of the
  platform in its current state is satisfactory. However, given the state of
  the code, it would probably be very hard for someone else to re-purpose the
  platform for needs somewhat outside of what was done here without requiring
  a significant re-write.

  The only remedy for this that comes to mind is more development experience in
  regards to this type of web related projects. It feels like it is safe to assume, that the
  majority of the target audience; people from the academia involved in
  usability studies, wanting to do remote tests, would, much like the author,
  lack extensive experience in coding this type of project and would likely
  fall into the same traps given a wide-open playing field. One way to offset
  this lack of experience somewhat would be to leverage the developer time put
  into a more a more opinionated web-framework like
  Django\cite{citeDjangoHomepage}, and, if needed change later when being more
  experienced and wanting something that is not currently achievable.

  %In this section there is a discussion about how the development process went
  %overall. Which parts of the process went ok and was there anything could have
  %been improved or tweaked?

  %\begin{itemize}
  %  \item{Discuss the process. Good bad? Done anything differently?}
  %\end{itemize}

\section{Deployment and gathering participants}

  Deployment of the platform succeeded without any problem, and giving
  participants access to the platform by sharing an url-address was very
  efficient.

%  It is worth noting that sharing the
%  link on Facebook was the only distribution method planned for initially.
%  After a suggestion that it would probably be interesting to sperad it

\section{Results}

  \begin{itemize}
    \item{Discuss the results.}
    \item{Anything weird?}
    \item{Everything as planned?}
    \item{Outliers, why?}
  \end{itemize}

\section{Possible improvements}

  What could be improved if this was done again?

  \subsection{Multiple design iterations}

    Should have had multiple smaller iterations, easy to get stuck.

  \subsection{Opt-in followup}

    Some questions reveled answers that didn't match what I had
    anticipated, would be interesting to at least have a chance to follow
    up on those observations.

  \subsection{Leverage more frameworks?}

    Didn't have the foundational expertise to decide on a good
    framework, would be easier now.
    Build upon frameworks for faster iteration?

\section{Threats to validity}

  \subsection{Online testing and latency}

  Latency should be minimal, but it would be better to measure and
  verify than simply guess.

  \subsection{Seeding with user-id}

  Might be a quirk in that re-seeding with the same user-id could
  possibly create 'easy' anonymous id's.

  Not possible to check screen-size and input method.

  Could have defined buttons better on the print-outs.

%				network latency?
%				multiple runs with same person?
