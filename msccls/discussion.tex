
This section reflects on the general processes underpinning the project as a
whole and any improvements or special considerations that have come to light
during its execution and completion.

\section{The design process}

  Having the paper prototypes as an initial lo-fi starting point and conducting
  the interviews to determine which to use worked very well, and is an element
  that should be repeated if a similar process would be done in the future.

  That being said, it is regrettable that the iterative design process did not
  have a chance to get into gear and display its full potential over multiple
  design runs. Only managing to get through one
  design-implement-release-and-test-cycle really hamstrung the possibility of
  experience positive effect of the process of dynamically evolving the
  platform over time in response to user feedback.

  \todo{Jag tycker att det är ett snyggt testupplägg men att det som testas är
  lite tunt, och kanske finns inte tillräckligt med info där för att testaren
  skall kunna göra denna rätt avanserade testning. Detta behöver du inte ändra
  något kring här men kan beröras i diskussionen när du diskuterar din metod.}

  \subsection{Adherence to the key principles of UCSD}

\section{The development process}

  It was very easy to set up and get starting working on creating the platform
  both with Python and Flask. In regard to the final performance and
  functionality of the platform in its current state is satisfactory for the
  indented use.

  However, given the state of the code, it would probably be very hard for
  someone else to re-purpose the platform for needs somewhat outside of what
  was done here without requiring a significant re-write.

  The only remedy for this that comes to mind is more development experience in
  regards to this type of web related projects. It feels like it is safe to
  assume, that the majority of the target audience; interested individuals
  inside of the gaming-industry, wanting to do remote tests, would, much like
  the author, lack extensive experience in coding this type of project and
  would likely fall into the same traps given a wide-open playing field.

  One way to offset this lack of experience somewhat would be to leverage the
  accumulated developer time put into a more opinionated web-framework like the
  aforementioned Django\cite{citeDjangoHomepage} project.

\section{Deployment and gathering participants}

  Deploying the platform on a local server exposed to the internet was done
  without much hassle, and it was easy enough to gather participants by
  sharing a url-link to the application.

  In hind-sight, the data gathering would benefit from having a more robust
  mechanic in place in order to differentiate if a significant portion of the
  participants came from a specific location, as an example, Massive.

  Since the initial distribution-plan did not include the internal Massive
  mailing list, the platform did not have this kind of mechanic in place, witch
  made int impossible to definitely split the final set of participants along
  that grouping if needed.

\section{Results}

  Event though the data gathered from the tests was rudimentary and the
  design of the test tasks as basic as possible they were still enough to
  determine some interesting characteristics about the different represented
  cases and how users interacted with them.

  \todo{Diskutera gärna hur du tror att testning av fler saker skulle påverka
  svarsfrekvens, och hur testupplägget skulle kunna utvecklas för att undvika
  att folk inte orkar svara.}

  Given more time and resources it would be trivial to expand the system to
  perform more multi-faceted data gathering together with more intricately
  design tasks in order to test specific usability target beyond
  time-to-completion.

  Overall, the size of the response, and how easy it was to run large batches
  of remote test and extract data from them with little to no experience was
  surprising. Especially since the data was only collected during five days
  before focus was shifted to processing said collected data.

%\section{Initial idea and pivot}
%
%  \todo{Re-write for this section} \\
%  It has been established that the output of a team benefits from information
%  sharing\cite{%
%  citeInformationSharingTeamPerformanceMeta,%
%  citeContextualizedRelationshipBetweenKnowledgeSharingandPerformanceinSoftwareDevelopment%
%  }
%  , and that such vital sharing tends to break down when it is needed
%  the most\cite{citeInformationSharingTeamPerformanceMeta}.
%
%  This effect is something that has been observed first hand by one of the
%  managers at Massive. Working with a team under pressure, nearing a deadline,
%  the software used for planing and information distribution was abandoned, in
%  favor for post-it-notes on a whiteboard. While acknowledging that it is a
%  method that works for smaller teams, the manager observed that it put a big
%  damper on the bandwidth available for information sharing among the
%  team-members.
%
%  Asking some of the employees why they opted to not use the software in the scenario
%  mentioned above, they answered that they felt it got in their way, hindering
%  them from doing their work, a clear usability issue.
%
%  One item that got a lot of attention in both sessions was the tendency of
%  information-sharing breaking down when it is needed the most. On the team
%  side, the discussions centered on the negative impact on team performance and
%  general well-being of the team members. The academic part of the discussion
%  revolved mainly around how this negative impact could be mitigated or
%  eliminated by improving the design and usability of the underlying
%  communication systems.
%
%  Initially the idea was to incrementally perform alterations to the existing
%  design of the communication system used, and then perform usability testing in
%  order to determine which, if any of the alterations would reduce the friction
%  of use during high-stress situation. But after an initial investigation
%  revealed that doing these kind of alteration on the current system would be
%  very hard, the idea was scrapped.


\section{Possible improvements}

  Even though the deployed platform impressed with the ease it collected
  response data, there are some considerations that should be taken into
  account if this process, or similar was to be repeated.

  \subsection{Let go early, fit multiple design iterations}

  In the gaming industry there is a commonly performed practice referred to as
  ``creating a vertical slice''. And though there does not seem to exist an
  official recording of its origin, or agreed upon definition, the different
  variant heard by the author are similar enough that the general understanding
  of what the process strives to do will be stated in the context of this
  report as follows:

  When producing a vertical slice, the goal is to create, with the least effort
  possible, something that engages the whole set of interconnected systems that
  will appear in the finalized project or product in order to verify that it is
  a viable effort as early as possible.

  Viewing the development of this platform through this lens, the initial vertical
  slice should have consisted of an non-styled html-page with a single input
  field and submit button connected to a database, accessible by a url-link
  that should have been distributed for feedback and iteration within the first
  days of starting development.

  \subsection{Opt-in follow-up}

  From its inception, this project has had an clear goal of keeping the
  participants as anonymous as possible and only collect the bare necessities
  to perform basic analysis.

  However, after processing the data, there are some answers that would have
  been interesting if they could have been followed up on. The most plausible
  solution while still keeping the anonymity of the participants would be some
  kind of option for a opt-in follow-up through anonymized email communication
  or similar.

  \subsection{Investigate and leverage frameworks}

  More time should have been spent investigating different available
  frame-works, both for the front-end and back-end of the platform development,
  given the plethora of available web-related solutions and frameworks available.

  Assuming one is not already experienced, there is much to be gained by
  adopting one a frameworks with this type of project, even if, after gaining
  some experience, the initially chosen framework is switched out for a better
  fit.

  \section{Threats to validity}

  This section highlights any known discrepancies related to the theory,
  implementation or execution of the project that could introduce threats to
  any results based on the collect data.

  \subsection{Online testing and latency}

  Since these tests are performed online, there is always the possibility that
  the quality of the connection, or rather lack thereof, could influence the
  measurements collected by the platform.

  Further iterations on similar type of timed testing should integrate some
  type of analysis of the connection quality established with the participant
  in question over a set period of time. Even though it would be hard to
  mitigate the effect of connection-influence, having access to this type data
  would make it possible to at least generate a confidence interval in regards
  to the registered values.

  \subsection{Default age value}\label{label_validity_default_age}

  \todo{Skulle kunna ligga under 5.5 istället eller?}

  Analyzing the returned data from the test pre-questionnaire, the most common
  age was twenty-five, which coincidentally was the precise value of the only
  pre-filled input field in any of the questions. Going forth, non of the
  fields should have a default value in order to avoid this kind of bias.

  \subsection{Users participating multiple times}

  Even though multiple test-runs with the same user beyond the initial five
  test-task runs was encouraged, running several tests as different users was
  not. Doing some basic correlation of the logs on the server running the
  platform, it seems that at least a handful of participants ran tests as
  different user-ids. Since this could introduce some unexpected biases to the
  collected data, it would be interesting to add some kind of flagging
  mechanism for when many different users appear to be coming from the same
  source, while still keeping it anonymous.
