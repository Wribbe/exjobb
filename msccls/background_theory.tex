%The main goal is to produce a system that has high usability. \todo{Expand}

Though there exists many definitions of the term usability, the author finds the
following one from the ISO Standard 9241-11:2018\cite{citeISO9241}, section 3.1
- Terms and definitions to be one of the more concise and direct ones.
\begin{quote}
  \textbf{usability} \\
  extent to which a system, product or service can be used by specified users to
  achieve specified goals with effectiveness, efficiency and satisfaction in a
  specified context of use
\end{quote}

In other words, for something to be usable, it has to help its user to
reach a goal, not only in an effective and efficient way, but it should also be
a satisfactory process to get there.

\section{User-centered design}

In 1986 Donald A. Norman contributed to the book
\citetitle{citeUserCenteredNorman}\cite{citeUserCenteredNorman} with, among
other things, a chapter titled Cognitive Engineering\cite[p.
31]{citeUserCenteredNorman}. Said chapter begins with the following prologue,
quoted in part below:

\begin{quote}
  \textit{Cognitive Engineering}, a term invented to reflect the enterprise I find
  myself engaged in: neither Cognitive Psychology, nor Cognitive Science, nor
  Human Factors. It is a type of applied Cognitive Science, trying to apply
  what is known from science to the design and construction of machines \ldots
  %of machines. It is
  %a surprising business. On the one hand, there actually is quite a lot known
  %in Cognitive Science that can be applied. But on the other hand, our lack of
  %knowledge is appalling. On the one hand, computers are ridiculously difficult
  %to use. On the other hand, many devices are difficult to use---the problem is
  %not restricted to computers
  there are fundamental difficulties in
  understanding and using most complex devices. So the goal of Cognitive
  Engineering is to come to understand the issues, to show how to make better
  choices when they exist, and to show what the trade-offs are \ldots
 % when, as is the
 % usual case, an improvement in one domain leads to deficits in another.
\end{quote}

He then continues with
specifying that he has two major goals. First, to understand the fundamental
principles behind human performance and action, specifically how they can be
used to further develop engineering-principles of design.
Second, while efficiency, power and ease-of-use are all desirable traits for
any system, they should not be used as the primary metric. Instead, he posits
that what systems should strive for most of all is to be pleasant, possibly
even fun to interact with.

Making note of the second goal, which could just as well have been an expansion
of the satisfactory aspect in the above mentioned usability definition. The
text then continues with
\textit{Analysis of Task Complexity},
\textit{The Gulfs of Execution and Evaluation} and
\textit{Stages of User Activities}, to name a few topics, and ends with
a section titled \textit{Prescriptions for Design Principles}.

In this final section he expands on the purposed functions and
accomplishments that should come as an result of developing Cognitive
Engineering further.
Considering the time the text was written, due to the lack of available
information on what fosters good interactions between people and machines,
there is no guidance given for the specific details of the design. There are
however, guiding prescriptions for how the design process might proceed\cite[p.
59-61]{citeUserCenteredNorman}, each with its own heading and describing
paragraph, headings replicated below:

\begin{itemize}
 \item{Create a science of user-centered design}
 \item{Take interface design seriously as an independent and important problem}
 \item{Separate the design of the interface from the design of the system}
 \item{Do user-centered system design: Start with the needs of the user}
\end{itemize}

Here, the last heading and its paragraph holds special interest since it lays
out the fundamentals of user-centered design as follows. User-centered design
should focus on serving the needs of the user above all else. While building a
system, the users needs should be the main driver behind any interface design,
and, in turn, the needs of said interface should then dictate the design and
composition of the remaining system.

A more recent touch-stone for user-centric design comes from the
\citeyear{citeKeyPrinciplesUserCentric} paper titled
\citetitle{citeKeyPrinciplesUserCentric}\cite{citeKeyPrinciplesUserCentric}.
Here, the authors argue that user-centered system design (UCSD) lacks an agreed
upon definition, and that the concept as a whole suffers for it. Using earlier
defined concepts from literature in combination with their own research and
software development experience the authors posit that UCSD should be seen as a
development process where the main goal is to focus on usability, not only
during development but also during the subsequent life cycle of the system.

Further, the authors provide twelve guiding
principles\cite[p. 401]{citeKeyPrinciplesUserCentric}
for any project using this definition of UCSD, with the first six being
especially relevant in the context of this report:


\begin{description}
  \item[User focus]{%
    The goals of the activity, the work domain or
    context of use, the users goals, tasks and  needs should early guide
    the development.%
  }
  \item[Active user involvement]{%
    Representative users should actively
    participate, early and continuously throughout the entire development
    process and throughout the system life cycle.%
  }
  \item[Evolutionary systems development]{The systems development should be
  both iterative and incremental%
  }
  \item[Simple design representations]{%
    The design must be represented in
    such ways that it can be easily understood by users and all other
    stakeholders.%
  }
  \item[Prototyping]{%
    Early and continuously, prototypes should be used to
    visualize and evaluate ideas and design solutions in cooperation with
    the end users.%
  }
  \item[Evaluate use in context]{%
    Baseline usability goals and design
    criteria should control the development.%
  }
\end{description}

The remaining six principles focus more on how to approach UCSD in a
corporate setting with larger teams by assigning usability team-roles and
having explicit design activities, neither of which is applicable in the
current development situation.

\todo{Reflect on how these were met in results}


\section{Usability testing}

  In \citeyear{citeHeuristicsNielsenUsabilityEngineering}, Jacob Nielsen wrote a
  book titled \citetitle{citeHeuristicsNielsenUsabilityEngineering}%
  \cite{citeHeuristicsNielsenUsabilityEngineering}, where he, among other
  things, builds on and clarifies previous work surrounding usability testing.
  In the preface he states the goal of the book as follows:

  \begin{quote}
    The main goal of the book is to provide concrete advice and
    methods that can be systematically employed to ensure a high
    degree of usability in the final user interface. To arrive at the perfect
    user interface, one also needs genius, a stroke of inspiration, and
    plain old luck. Even the most gifted designers, however, would be
    pressing their luck too far if they were to ignore systematic usability
    engineering methods.
  \end{quote}

  This quote, from the introduction to chapter 6,
  \textit{Usability Testing}%
  \cite[p. 165]{citeHeuristicsNielsenUsabilityEngineering}
  does a good job exemplifying one of the core tenants underpinning both
  usability testing and user centered design.

  \begin{quote}
    User testing with real users is the most fundamental usability method and
    is in some sense irreplaceable, since it provides direct information about
    how people use computers and what their exact problems are with the
    concrete interface being tested.
  \end{quote}

  Continuing the chapter, Nielsen introduces the following concepts, probably
  familiar to anyone working with usability testing:
  \textit{Test Gaols and Test Plans},
  \textit{Getting Test Users},
  \textit{Choosing Experiments},
  \textit{Test Tasks},
  \textit{Stages of a Test},
  \textit{Performance Measurement},
  \textit{Thinking Aloud} and
  \textit{Usability Laboratories}.
  Looking at more resent literature related to usability testing, it is not
  hard to identify concepts that are, if not identical, at least similar to the
  ones mentioned above.

  In the \citeyear{citeHandbookUsability} edition of
  \citetitle{citeHandbookUsability}\cite[p. 19]{citeHandbookUsability}, the
  authors talk about usability testing being roughly split into two approaches.
  The first being the more formal one, where tests are conducted as true
  experiments, with the goal to confirm or refute one or more specific
  hypothesises. The second testing approach, which book focuses on, is
  characterized by an iterative cycle of tests and is somewhat less formal, but
  still rigorous. Core to this type of testing is the iterative testing cycle,
  intended to gradually expose shortcomings in the design of the system under
  test and continuously re-shape it based on collected data and evaluations
  from the performed tests.

  It is this second iterative type of usability testing that is the basis for
  any usability testing conducted as part of this report. Also, if not
  specified otherwise, any further mentions of \textit{usability testing}
  refer to this type of iterative approach.
