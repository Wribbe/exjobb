\section{Usability}

  \subsection{Why usability testing?}

    Initially, the plan was to make interface changes to the
    organization-software itself, but a question remained, how do you prove
    that it actually makes a difference?

    \todo{Describe why usability testing, add scaling problem.}

  \subsection{Introduction}

    Usability is traditionally done in person with the \textit{over the
    shoulder} method which gives a good insight of what a participant does
    during a test. Further more, if the \textit{thinking out aloud -
    method} is utilized correctly, the test-moderator should have a good
    insight into the participant thought-process during the test.

  \subsection{Evolution and current use}

    While effective\checkTruth this method scales poorly with a one to one
    ratio between moderator and test participant. This project investigates
    the possibility of alleviating this scale constraint by utilizing a
    internet based platform to conduct usability tests of user interfaces
    online.

    \todo{Expand this section.}


\section{Python and Flask}

  The backbone of the testing-platform is written in  Python\cite{citePython},
  a general purpose programming language that is steadily becoming
  more and more popular among developers according to the 2019 installment of
  the annual developer survey\cite{citeStackOverflow2019Survey},
  conducted by popular programming questions and answers site,
  StackOverflow\cite{citeStackOverflow}.

  In addition to the author having experience with Python, the language was
  chosen due to its interesting relation to both data-analysis and
  web-development. When
  JetBrains\cite{citeJetBrains},
  creators of
  PyCharm\cite{citePyCharm},
  a Python coding environment, asked 24 000 Python
  developers:
  ``What do you use Python for?''\cite{citeJetSurvey}, allowing for multiple
  selections, 59\% answered data analysis and 51\% web development.

  Python is also the home of SciPy, ``a Python-based ecosystem of open-source
  software for mathematics, science and engineering''\cite{citeSciPyHomepage},
  that has become ``a de facto
  standard for leveraging scientific algorithms in
  Python''\cite{citeSciPyPaper}, making the language a good fit for processing
  data from studies in various ways.

  In order to have any data to analyse there needs to be some kind of
  web-component added to the mix. The choice came down to the following two
  popular Python web-frameworks, Flask\cite{citeFlaskHomepage} and
  Django\cite{citeDjangoHomepage}. Django facilitates an established structure
  and more features defined out-of-the-box, also known as ``batteries
  included'', while Flask encourages a more build it as you need it approach.

  In the end Flask was chosen because it seemed to better support an iterative
  development process, and it was hard to know if the project-structure enforced
  by Django would be a good fit at the beginning of this process.


%  \cite{citeIeeeSpectrum}
%
%  \cite{citepypl}
%
%  \cite{citetiobe}

\cite{citeBeyondTheUsabilityLab}
\cite{citeBenefitsRemote}
\cite{citeBottomLine}
\cite{citeImpactSoftwareDesign}
\cite{citeKeyPrinciplesUserCentric}

\section{Scalable Vector Graphics}

\section{To JavaScript or not to JavaScript}

%https://businessoverbroadway.com/2019/01/13/programming-languages-most-used-and-recommended-by-data-scientists/
%https://www.jetbrains.com/lp/python-developers-survey-2019/
