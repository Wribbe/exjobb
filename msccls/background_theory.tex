The main goal is to produce a system that has high usability. \todo{Expand}

Though there exists many definitions of the term usability, the author finds the
following one from the ISO Standard 9241-11:2018\cite{citeISO9241}, section 3.1
- Terms and definitions to be one of the more concise and direct ones.
\begin{quote}
  \textbf{usability} \\
  extent to which a system, product or service can be used by specified users to
  achieve specified goals with effectiveness, efficiency and satisfaction in a
  specified context of use
\end{quote}

In other words, for something to be usable, it has to help its user to
reach a goal, not only in an effective and efficient way, but it should also be
a satisfactory process to get there.

%\begin{quote}
%HCI draws on many disciplines, as we shall see, but it is in computer science and
%systems design that it must be accepted as a central concern.
%\dots
%%For all the other discip-
%%lines it can be a specialism, albeit one that provides crucial input;
%for systems design it is an essential part of the design process. From this
%perspective, HCI involves the design, implementation and evaluation of
%interactive systems in the context of the user’s task and work.
%\end{quote}
%
%Human Computer Interaction (HCI) is the
%
%How can you know if a design is usable or not?

%Initially the idea was to streamline the currently used
%production/communication tool and measure the usability impacts, however,
%it was not possible to make direct changes to the tools appearance at the time.
%Instead there was a pivot to figuring out how to construct a self-hosted system
%for usability measurements.

\section{User-centered design}

In 1986 Donald A. Norman contributed to the book
\citetitle{citeUserCenteredNorman}\cite{citeUserCenteredNorman} with, among
other things, a chapter titled Cognitive Engineering\cite[p.
31]{citeUserCenteredNorman}. Said chapter begins with the following prologue,
quoted in part below:

\begin{quote}
  \textit{Cognitive Engineering}, a term invented to reflect the enterprise I find
  myself engaged in: neither Cognitive Psychology, nor Cognitive Science, nor
  Human Factors. It is a type of applied Cognitive Science, trying to apply
  what is known from science to the design and construction of machines \ldots
  %of machines. It is
  %a surprising business. On the one hand, there actually is quite a lot known
  %in Cognitive Science that can be applied. But on the other hand, our lack of
  %knowledge is appalling. On the one hand, computers are ridiculously difficult
  %to use. On the other hand, many devices are difficult to use---the problem is
  %not restricted to computers
  there are fundamental difficulties in
  understanding and using most complex devices. So the goal of Cognitive
  Engineering is to come to understand the issues, to show how to make better
  choices when they exist, and to show what the tradeoffs are \ldots
 % when, as is the
 % usual case, an improvement in one domain leads to deficits in another.
\end{quote}

He then continues\cite[p. 32]{citeUserCenteredNorman}:
\begin{quote}
  \ldots Overall, I have two major goals:
  \begin{enumerate}
    \item{To understand the fundamental principles behind human action and performance that are relevant for the
    development of engineering principles of design.}
    \item{To devise systems that
    are pleasant to use--the goal is neither efficiency nor ease nor power, although
    these are all to be desired, but rather systems that are pleasant, even fun
    \ldots}
  \end{enumerate}
\end{quote}

Making note of the second goal, which could just as well have been an expansion
of the satisfactory aspect in the above mentioned usability definition. The
text then continues with
\textit{Analysis of Task Complexity},
\textit{The Gulfs of Execution and Evaluation} and
\textit{Stages of User Activities}, to name a few topics, and ends with
a section titled \textit{Prescriptions for Design Principles}.

This last section expands on what purposed functions and
accomplishments should come as an result of developing Cognitive Engineering
further.
At the time of writing, due to the lack of available information on what
fosters good interactions between people and machines, there
are no guidance given for the specific details of the design. There are however,
guiding prescriptions for how the design process might proceed\cite[p.
59-61]{citeUserCenteredNorman}, each with its own heading and describing
paragraph, headings replicated below:

\begin{itemize}
 \item{Create a science of user-centered design}
 \item{Take interface design seriously as an independent and important problem}
 \item{Separate the design of the interface from the design of the system}
 \item{Do user-centered system design: Start with the needs of the user}
\end{itemize}

It is the last heading and accompanying paragraph that is of special interest
in this context since it contains the following definition of what
user-centered design should strive for\cite[p. 59-61]{citeUserCenteredNorman}:

\begin{quote}
  user-centered design emphasizes that
  the purpose of the system is to serve the user, not to use a
  specific technology, not to be an elegant piece of programming.
  The needs of the users should dominate the design of the inter-
  face, and the needs of the interface should dominate the design
  of the rest of the system.
\end{quote}

A more recent touch-stone for user-centric design comes from the
\citeyear{citeKeyPrinciplesUserCentric} paper titled
\citetitle{citeKeyPrinciplesUserCentric}\cite{citeKeyPrinciplesUserCentric}.
Here, the authors argue that user-centered system design (UCSD) lacks an
agreed upon definition, and that the concept as a whole suffers for it.
Building on earlier defined concepts from literature combined with their own
research and software development experience, they present the following
definition of UCSD\cite[p. 401]{citeKeyPrinciplesUserCentric}:

\begin{quote}
  User-centred system design (UCSD) is a process focusing on usability
  throughout the entire development process and further throughout the system
  life cycle. \ldots based on the following [12] key principles:
\end{quote}

Headings and abbreviated descriptions of the key principles below:

\begin{itemize}
  \item{%
      \textit{%
        User focus -- the goals of the activity, the work domain or
        context of use, the users' goals, tasks and  needs should early guide
        the development%
      }\cite{citeKeyPrinciplesUserCentric}

      It is critical that all members of a team understand who the users are,
      what situation they are in and what goals they are trying to achieve.
  }
  \item{%
      \textit{%
        Active user involvement -- representative users should actively
        participate, early and continuously throughout the entire development
        process and throughout the system lifecycle%
      }\cite{citeKeyPrinciplesUserCentric}
  }
  \item{%
      \textit{%
        Evolutionary systems development -- they systems development should be
        both iterative and incremental%
      }\cite{citeKeyPrinciplesUserCentric}
  }
  \item{%
      \textit{%
        Simple design representations -- the design must be represented in
        such ways that it can be easily understood by users and all other
        stakeholders%
      }\cite{citeKeyPrinciplesUserCentric}
  }
  \item{%
      \textit{%
        Prototyping -- early and continuously, prototypes should be used to
        visualize and evaluate ideas and design solutions in cooperation with
        the end users%
      }\cite{citeKeyPrinciplesUserCentric}
  }
  \item{%
      \textit{%
        Evaluate use in context -- baselined usability goals and design
        criteria should control the development%
      }\cite{citeKeyPrinciplesUserCentric}
  }
  \item{%
      \textit{%
        Explicit and conscious design activities -- the development process
        should contain dedicated design activities%
      }\cite{citeKeyPrinciplesUserCentric}
  }
  \item{%
      \textit{%
        A professional attitude -- the development process should be performed
        by effective multidisciplinary teams%
      }\cite{citeKeyPrinciplesUserCentric}
  }
  \item{%
      \textit{%
        Usability champion -- usability expert should be involved early and
        continuously throughout the development lifecycle%
      }\cite{citeKeyPrinciplesUserCentric}
  }
  \item{%
      \textit{%
        Holistic design -- all aspects that influence the future use situation
        should be developed in parallel%
      }\cite{citeKeyPrinciplesUserCentric}
  }
  \item{%
      \textit{%
        Process customization -- the UCSD process must be specified, adapted
        and/or implemented locally in each organization%
      }\cite{citeKeyPrinciplesUserCentric}
  }
  \item{%
      \textit{%
        A user-centered attitude should always be established%
      }\cite{citeKeyPrinciplesUserCentric}
  }
\end{itemize}
\todoMaybe{restructure, should this be used?}



%Here, the authors argue that since it was coined by Norman and Draper in 1986,
%user-centered system design (UCSD)
%since it was first coined by Norman and Draper in 1986, suffered from having
%competing and differing . and is, in practice becoming a concept with no meaning.


\section{Usability testing}

%  Initially, the plan was to make interface changes to the
%  organization-software itself, but a question remained, how do you prove
%  that it actually makes a difference?
%

  In \citeyear{citeHeuristicsNielsenUsabilityEngineering}, Jacob Nielsen wrote a
  book titled \citetitle{citeHeuristicsNielsenUsabilityEngineering}%
  \cite{citeHeuristicsNielsenUsabilityEngineering}, where he, among other
  things, builds on and clarifies previous work surrounding usability testing.
  In the preface he states the goal of the book as follows:

  \begin{quote}
    The main goal of the book is to provide concrete advice and
    methods that can be systematically employed to ensure a high
    degree of usability in the final user interface. To arrive at the perfect
    user interface, one also needs genius, a stroke of inspiration, and
    plain old luck. Even the most gifted designers, however, would be
    pressing their luck too far if they were to ignore systematic usability
    engineering methods.
  \end{quote}

  This quote, from the introduction to chapter 6,
  \textit{Usability Testing}%
  \cite[p. 165]{citeHeuristicsNielsenUsabilityEngineering}
  does a good job exemplifying, what this author thinks, is one of the
  core tenants underpinning both usability testing and user centered
  design.

  \begin{quote}
    User testing with real users is the most fundamental usability method and
    is in some sense irreplaceable, since it provides direct information about
    how people use computers and what their exact problems are with the
    concrete interface being tested.
  \end{quote}

  Continuing the chapter, Nielsen introduces the following concepts, probably
  familiar to anyone working with usability testing:
  \textit{Test Gaols and Test Plans},
  \textit{Getting Test Users},
  \textit{Choosing Experiments},
  \textit{Test Tasks},
  \textit{Stages of a Test},
  \textit{Performance Measurement},
  \textit{Thinking Aloud} and
  \textit{Usability Laboratories}.
  Looking at more resent literature related to usability testing, it is not
  hard to identify concepts that are, if not identical, at least similar to the
  ones mentioned above.

  In the \citeyear{citeHandbookUsability} edition of
  \citetitle{citeHandbookUsability}\cite[p. 19]{citeHandbookUsability}, the
  authors talk about usability testing being roughly split into two approaches.
  The first being the more formal one, where tests are conducted as true
  experiments, with the goal to confirm or refute one or more specific
  hypothesises. The second testing approach, which book focuses on, is
  characterized by an iterative cycle of tests and is somewhat less formal, but
  still rigorous. Core to this type of testing is the iterative testing cycle,
  intended to gradually expose shortcomings in the design of the system under
  test and continuously re-shape it based on collected data and evaluations
  from the performed tests.

  It is this second iterative type of usability testing that is the basis for
  any usability testing conducted as part of this report. Also, if not
  specified otherwise, any further mentions of \textit{usability testing}
  refer to this type of iterative approach.



%  \todo{Describe why usability testing, add scaling problem.}

  %\subsection{Introduction}

  %  Usability is traditionally done in person with the \textit{over the
  %  shoulder} method which gives a good insight of what a participant does
  %  during a test. Further more, if the \textit{thinking out aloud -
  %  method} is utilized correctly, the test-moderator should have a good
  %  insight into the participant thought-process during the test.

%  \subsection{Evolution and current use}
%
%    While effective\checkTruth this method scales poorly with a one to one
%    ratio between moderator and test participant. This project investigates
%    the possibility of alleviating this scale constraint by utilizing a
%    internet based platform to conduct usability tests of user interfaces
%    online.
%
%    \todo{Expand this section.}

  \subsection{Why bother with usability?}

    Usability and the bottom line
    \cite{citeBottomLine}

    Analysing the impact of usability on software design
    \cite{citeImpactSoftwareDesign}


