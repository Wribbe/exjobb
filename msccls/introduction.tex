
%
%In the 2019 their yearly publication \cite{citeDeveloperIndex}. grew with
%48\% revenue
%grew with 33\% to a total of 1.87 billion EUR in 2018.
%
%
%As the gaming industry continue to grow{\findref\findref} so do the reported
%number of stress-related issues reported by the people working in the
%sector{\findref\findref\findref}.


%The initial idea for this report came from a managers
%observation that co-workers would abandon the digital communication software
%for a more hands-on approaches, such as post-it notes on a whiteboard, when
%the pressure got to a certain point.
%
%Asking why, people stated that the software they were supposed to use for
%communicating and propagating the projects status throughout the team got in
%their way. Which is why they opted to use post-it notes, even though it has
%significantly worse communication bandwidth and is less accessible,
%at-least-it-works\texttrademark.
%
\todo{Hur intressant detta spår än är så får du lyfta bort det så att mål och
syfte blir tydligt genom hela rapporten.}


\section[MASSIVE Entertainment | A Ubisoft studio]{MASSIVE}

{\vspace{-0.7cm}{\hspace{1.85cm}\small MASSIVE ENTERTAINMENT | A UBISOFT STUDIO}}

  Massive Entertainment is a world-leading game-development-studio, founded by
  Martin Walfisz in 1997 and located in Malmö, Sweden.

  Since their first release, Ground Control in June of 2000,
  the company has produced a string of critically acclaimed games,
  and has been part of both Vivendi Universal Games and Activation Blizzard. As
  of 2008 they are a permanent part of Ubisoft, a video-games company
  with its headquarter based in Montreuil, France with associated studios all
  over the world.

  After becoming a part of the Ubisoft family, Massive managed to break
  Ubisofts record for most copies sold in 24 hours with the release of Tom
  Clancy's The Division on the 8'th of March 2016, which then set another
  record for having the biggest first week ever for a new game franchise.
  At the time of writing Massive has more than 650 employees working at their
  Malmö studio and has announced that their next big project is related to
  James Cameron's Avatar.

\section{The Swedish game-development sector}

  Since 2006, Dataspelsbranchen, a Swedish trade association for video game
  companies\cite{citeDataspelsbranchen}, has released a yearly report called
  `Spelutvecklarindex` where they gather and publish information related to the
  growth of Swedish game development companies.
  According to their most resent publication,
  `Spelutvecklarindex 2019`\cite{citeIndex2019}, the domestic Swedish
  game-development sector continues to grow steadily.

  In 2018, according to the report, the total revenue of the sector grew with
  33\%, totalling 1.87 billion EUR, with the number jobs provided also increasing
  with 14\%, totalling 5 320\cite[p. 12]{citeIndex2019} full-time positions,
  at 384\cite[p. 38]{citeIndex2019} active game-development companies in Sweden.

  With its 650-plus employees, Massive enters the statistics as the 4'th
  largest game-development studio in Sweden\cite[p. 21]{citeIndex2019},
  making them a big potential influence on how the sectors develops in the
  future, both in terms of their employees as well as the players of their
  games.


\section{Global game-development and usability}

  \todo{Intressant spår i sig. Återknyt till detta i diskussioner, dvs hur den
  lösning du presenterat i ditt arbete kan vidareutvecklas och användas för
  detta.}

%  The corresponding organization to `Dataspelbranchen`
  According to the Entertainment Software Association, or ESA, witch is the trade
  association for the video game industry in the United States the game-development
  is on a steady rise there to. In their latest report, titled `2019 Essential Facts`%
  \cite{cite2019EssentialFactsAbouttheComputerandVideoGameIndustryEntertainmentSoftwareAssociation}
  they write that 65\% of American adults play video
  games\cite[p. 4]{cite2019EssentialFactsAbouttheComputerandVideoGameIndustryEntertainmentSoftwareAssociation}
  , and information at their official homepage states that from 2017 to 2018
  the industry grew with 18\%, reaching a record-high \$43.4 billion dollars in
  revenue\cite{eseaEconomicGrowth}.

  As digital sales reaches 83\% of the total purchase volume in America%
  \cite[p. 20]{cite2019EssentialFactsAbouttheComputerandVideoGameIndustryEntertainmentSoftwareAssociation}
  game-development is becoming even more global and distributed. This trend affects
  not only the sale of the finished games, but also the development of said
  games, with 31 Swedish studious employing a total of 2 604 persons
  abroad\cite[p. 27]{citeIndex2019}.

  In this type of large, competitive and global market, the author thinks that
  usability could provide an interesting edge.

  In \citetitle{citeBottomLine}\cite{citeBottomLine} the author
  \citeauthor{citeBottomLine} argues that, while few people disagree with
  the notion that usability engineering benefits the end user, it is harder to
  convince people of the benefits for companies and its employees.

  And even though the piece does not focus on game-development directly, the
  general game-development procedures overlaps all the examples
  mentioned; IT system (backend-systems and servers for games), e-commerce (the
  digital sale of the game to consumers) and shrink-wrapped software (the actual
  distribution of the game binaries).

  This makes game-development an excellent candidate benefiting from almost all
  of the listed benefits of incorporating large-scale usability in a company,
  but most fittingly;
  \begin{itemize}
    \item{Reduced development and maintenance}
    \item{Lower support cost}
    \item{Improved productivity and efficiency}
    \item{Reduced training costs}
    \item{Reduced documentation costs}
  \end{itemize}

  It is also worth mentioning a specific quirk that gives the gaming-industry a
  slight upper hand compared to the rest of the software industry in regards to
  usability-testing. Even though it might be used slightly
  incorrectly\cite{citeThinkAloud} in comparison to its often cited
  source\cite{ProtocolanalysisVerbalReportsAsData}, the usage of `thinking
  aloud protocol' where ``subjects verbalize their thoughts at the time that
  they [surface]''\cite[p. 60]{ProtocolanalysisVerbalReportsAsData} is widely
  used, prompting Nielsen to state ``thinking aloud may be the single most
  valuable usability engineering method''\cite[p. 195]{citeHeuristicsNielsenUsabilityEngineering}

  Thanks to the current trend of streaming games to a invisible public, a
  company with a popular game being streamed will have high-definition videos
  of players playing their game, while providing commentary, available, for
  free. And while not providing the same insight and fidelity as a correctly
  performed in-person thinking-aloud-session, this material could be used to
  introduce the concept of thinking-aloud to a company, or as training
  material for new usability
  engineers\cite{citeYouTubeGamersandThinkAloudProtocolsIntroducingUsabilityTesting}.

\section{Goals}

  \todo{Jag tycker dessa är rätt så klara men som opponenterna påpekade, se
  till att följa upp, i slutet av rapporten på vilket sätt du lyckades uppnå
  dessa mål.}

  \begin{enumerate}
    \item{
      Create a web based platform for basic interface usability testing.
    }
    \item {
      The platform should be able to introduce and administer a group of
      potentially geographically distributed individuals to task-based
      usability-testing.
    }
    \item{
      Develop the platform by collecting feedback and user-statistics and using
      the data to inform where and how to make changes in order to improve the
      platform.
    }
  \end{enumerate}
%
%\section{Literary scope}
%
%  This report draws and builds on information from the fields of usability
%  testing, web design and interaction design.
%
%  Specifically, \ctitle{citeHandbookUsability} and
%  \ctitle{citeUsabilityTestingEssentials}, provide a contrast between
%  traditional and modern approaches to usability testing and how to perform
%  them.
%
%  \ctitle{citeDonMakeMeThink} provides a concise and interesting
%  summary of no-nonsense approaches to web design from a usability perspective.
%  Last but not least \ctitle{citeTheDesignOfEverydayThings} introduces
%  both \textit{user-centered-} and \textit{interaction-design} together
%  with the concept of \textit{affordances},
%
%  \todo{REDO WHOLE SECTION}
