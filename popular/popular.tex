\documentclass{article}
\usepackage[utf8]{inputenc}
\usepackage[swedish]{babel}
\usepackage[a4paper, margin=1.5cm]{geometry}
\usepackage{parskip}
\usepackage{multicol}
\usepackage{graphicx}
\setlength{\columnsep}{1cm}

%\usepackage[sfdefault]{roboto}  %% Option 'sfdefault' only if the base font of the document is to be sans serif

\usepackage[T1]{fontenc}
\usepackage{addfont}
\addfont{OT1}{tap}{\tapir}

\begin{document}

  \thispagestyle{empty}

%  \resizebox{.95\linewidth}{!}{Contribute to the development}\\
%  \hspace*{1cm}\resizebox{.9\linewidth}{!}{of your favorite game, from home!}
  \resizebox{.99\linewidth}{!}{Playing with untapped usability potential!}\\
  \hspace*{0.0cm}\resizebox{.95\linewidth}{!}{What usability-engineers and
  game-developers could learn from each other.}
  \\

  {\large
    This article wants to bring attention to what seems to be an untapped
    beneficial connection between the game-development sector and the sciences
    of usability-testing. First, studios that already develop games with an
    online component, especially if the game is popular, are probably sitting
    on a usability-testing goldmine. These studios could, with a little bit of
    added work and by pulling from the rich science of user- and
    usability-testing, potentially augment the testing already done in
    dedicated quality assurance labs with usability-data taken straight from
    the their end-users, the players. Inversely, there is potential to utilize
    insights from the game-development sector to help usability-engineers
    create and design test-environments that are both fun to engage with and
    help with participant retention, especially in cases where the complexity
    of the tests could scare away would be test participants.
  }
  \vspace{0.2cm}
  {
%    \hspace{5.5cm}
     \\
    \hspace*{\fill}
    \raisebox{0.15cm}{{\today} -- Stefan Eng}
  }
  \vspace{-0.3cm}
  \begin{multicols*}{2}[]
    What could fields fields of user testing and game-development learn from
    each other? More than initially meets the eye apparently! The first hint
    that there could exist an interesting connection between the two came after
    successfully completing the first phase of a student-development project at
    Massive Entertainment. The goal of the project was to see if it was
    possible for a single developer to create a basic web-application for
    running rudimentary user-tests over the internet using Python and Flask.

    It was during the subsequent online testing that some of the connected
    participants, that ideally should only do the set of tests once, came back
    for more, stating that they had so much fun they wanted to do it again!

  \subsection*{Massive Entertainment}

    Massive Entertainment is a world-leading game-development studio
    based in Malmö, Sweden. The studio has been a part of the Ubisoft family
    since 2008 and is currently the fourth largest studio in Sweden with its
    650\texttt{+} employees.

  \subsection*{A usability-testing primer}

    First, what do we mean when we say usability? The term commonly refers to
    how efficient and satisfactory a system, product or service is to use when
    interacting with it to complete a specific goal.

    Usability-testing then, is simply the process of trying to figure out the
    usability of something. The process typically involves at least two people;
    one test participant and one test moderator. The participants role is to
    interact with the system or product according to any stated instructions or
    goals, and the moderators is there to silently observe and take note of any
    problems encountered by the participant.

  \subsection*{The point of taking it all online}

    While the process in described in the previous section works very well, it
    usually involves one trained moderator for each test participant, which, as
    you can imagine, scales rather poorly if you want to have a larger volume
    of usability-tests done.

    Here is where the development-project, internet and connectivity comes in.
    The successful tests run on the project shows that by leveraging the
    internet and the associated web-based technologies it should be possible to
    conduct rigorous usability-testing on larger scales. And while it will not
    be as insightful as a well-seasoned testing moderator in a one-to-one
    session, there are a myriad of useful things a computer could track
    during the participants performances for further analysis and aggregation.

  \subsection*{Games and accidental usability-testing}

  \end{multicols*}

\end{document}
