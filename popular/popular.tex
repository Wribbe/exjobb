\documentclass{article}
\usepackage[utf8]{inputenc}
\usepackage[swedish]{babel}
\usepackage[a4paper, margin=1.5cm]{geometry}
\usepackage{parskip}
\usepackage{multicol}
\usepackage{graphicx}
\usepackage{lettrine}
\setlength{\columnsep}{1cm}

%\usepackage[sfdefault]{roboto}  %% Option 'sfdefault' only if the base font of the document is to be sans serif

\usepackage[T1]{fontenc}
\usepackage{addfont}
\addfont{OT1}{tap}{\tapir}

\begin{document}

  \thispagestyle{empty}

%  \resizebox{.95\linewidth}{!}{Contribute to the development}\\
%  \hspace*{1cm}\resizebox{.9\linewidth}{!}{of your favorite game, from home!}
  \resizebox{.99\linewidth}{!}{Playing with untapped usability potential}\\
  \hspace*{0.0cm}\resizebox{.95\linewidth}{!}{What usability-engineers and
  game-developers could learn from each other}
  \\

  {\large
    This article brings attention to what seems to be an untapped beneficial
    connection between game-development and the sciences of usability-testing.
    Are you part of a studio that has released a game with an online component?
    If that is the case, you might be just a few tweaks away from a
    usability-testing goldmine! Inversely, are you a usability-engineer
    frustrated with how cumbersome it is to gather and retain participants for
    your usability-testing studies? Why not use the connectivity of the
    internet to gather participants and take some design-pointers from a
    industry with years of experience on introducing people to complex systems
    and increase retention?
  }
  \begin{multicols*}{2}[]
    \lettrine{W}{hat} could the fields of user testing and game-development learn from each
    other? More than initially meets the eye apparently! The first hint that
    there could exist an interesting connection between the two came after
    successfully completing the first phase of a student-development project at
    Massive Entertainment. The goal of the project was to see if it was
    possible for a single developer to create a basic web-application for
    running rudimentary usability-tests over the internet using Python and
    Flask.

    It was during the subsequent online testing that some of the connected
    participants, that ideally should only do the set of tests once, came back
    for more, stating that they had so much fun they wanted to do it again!

  \subsection*{About Massive Entertainment}

    Massive Entertainment is a world-leading game-development studio
    based in Malmö, Sweden. The studio has been a part of the Ubisoft family
    since 2008 and is currently the fourth largest studio in Sweden with its
    650\texttt{+} employees.

  \subsection*{A usability-testing primer}

    First, what do we mean when we say usability? The term commonly refers to
    how efficient and satisfactory a system, product or service is to use when
    interacting with it to complete a specific goal.

    Usability-testing then, is simply the process of trying to figure out the
    usability of something. The process typically involves at least two people;
    one test participant and one test moderator. The participants role is to
    interact with the system or product according to any stated instructions or
    goals, and the moderators is there to silently observe and take note of any
    problems encountered by the participant.

  \subsection*{The point of taking it all online}

    While the process described in the previous section works very well, it
    scales rather poorly. It is possible for a skilled moderator to keep track
    of a small group of participants, but doing higher-volume tests beyond that
    becomes problematic.

    Given the successful run of the trial test-platform, the possibility
    of leveraging the internet and associated web-based technologies for
    usability-testing should be investigated further.
    And while such test will not be as insightful as a well-seasoned testing
    moderator in a one-to-one setting, there are a myriad of useful things a
    computer could track for further analysis and aggregation.

  \subsection*{Accidental usability-testing}

    Coming back to usability-testing, one of the most popular and
    well-known variants is the \textit{think aloud protocol}. In this setup,
    the participant is instructed to voice any thoughts they have during the
    interaction out aloud as they come up. Now, given the context of gaming,
    this might sound weirdly familiar for some.

    Anyone guessing this was headed towards game-streaming was right.
    Online-streaming of games has become a major part of the
    entertainment-sector, providing countless hours of hi-definition video- and
    audio-commentary of people interacting with intricate systems.

  \subsection*{Bridging the last gap}

    In conclusion, even the experience from a short web-based usability-testing
    project indicates that both game-developers and usability-testing engineer
    could benefit from becoming even more like each other than they
    accidentally already are.

  \\ \ \\
  {
%%    \hspace{5.5cm}
%     \\
    \hspace*{\fill}
    \raisebox{0.15cm}{{\today} -- Stefan Eng}
  }
%  \vspace{-0.3cm}

  \end{multicols*}

\end{document}
